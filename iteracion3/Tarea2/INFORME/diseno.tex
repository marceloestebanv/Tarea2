chapter{Diseño}
\label{capdiseno}

\section{Introducción}

El Informe 3 equivale al capítulo de Diseño de su Trabajo de Título.
 En este documento se describen los contenidos mínimos exigidos para el Informe 3 y se sugiere la extensión (en cantidad de páginas) de cada tópico.
Los contenidos  se definen según el tipo de trabajo que se está realizando: desarrollo de SW, investigación,  u otro.
 Sin embargo, es el profesor guía quién debe aprobar su organización definitiva.
 También se debe resguardar la calidad y confiabilidad de las fuentes bibliográficas.

\section{Diseño Trabajos de Título de Desarrollo de SW}  \label{diseno}
 En este capítulo s e describe los contenidos del Capítulo de diseño para los TT de desarrollo de software.
 
 Obs.: Resguardar la trazabilidad y consistencia entre los modelos. Justifique las
decisiones de diseño.

\subsection{Diseño arquitectónico} \label{disenoarq}
Defina (si corresponde) los patrones de dise~no que usará. Incluya modelo de estructura
del sistema el cual debe reflejar el tipo de arquitectura especifica (cliente-servidor,
3 capas, SOA). Cada módulo debe estar trazado con respecto a los subsistemas identificados en el modelo de estructura.
De ser necesario incluya modelo de control.
Extensión máxima sugerida 5 páginas.


\subsection{Diseño de interfaz} \label{disenoint}
Sobre la base del perfil de usuario debe seleccionar el estilo de interacción, definir
los objetivos de facilidad de uso, determinar las pautas de estilo. Incluya en anexo el
modelo de navegacióon. Extensión máxima sugerida 5 páginas.

\subsection{Diseño lógico} \label{disenolog}
De acuerdo con  la metodología seleccionada, especifique los modelos de dise~no requerido,
por ejemplo para orientación a objeto casos de uso reales (en anexo), diagramas de
colaboración (en anexo) y diagramas de clases. Extensión máxima sugerida 15 páginas.


\subsection{Diseño de datos}  \label{disenodat}
A partir de cada subsistema (consistente con el modelo de estructura del sistema),
definir una componente ER con notación Bachmann. Defina claves candidatas, entidades,
cardinalidades y atributos. Normalice y justifique la normalización. Integre los
componentes ER en un modelo de datos lógico interno global Bachmann identificando
claves primarias, foráneas, atributos, tipos de relaciones y cardinalidades. Extensión máxima sugerida 5 páginas.


 \subsection{Diseño de pruebas}

  Debe definir cuidadosamente el objetivo y como realizará las pruebas de cada parte de su desarrollo:  Pruebas de requerimientos, Pruebas de análisis,  Pruebas de diseño, Pruebas de unidad, Pruebas de integración. Pruebas de sistema. Pruebas de aceptación del usuario, entre otras.


\subsection{Conclusiones}  \label{conclusiones}
Incluya análisis crítico sobre  pertinencia del problema, solución propuesta, proyecciones
y estado de avance. Extensión máxima sugerida 2 páginas.


\section{Diseño para Trabajos de Título de Investigación}
\label{diseno}
En este capítulo s e describe los contenidos del Capítulo de diseño para los TT de Investigación
\subsection{Diseño de la solución} \label{disenosol}
Defina brevemente el contexto de la solución propuesta. Luego defina detalladamente
la solución. Para esto debe incluir modelos o diagramas descriptivos globales
y luego incluir información detallada sobre cada módulo. En el caso de definición de
modelos, se deben incluir variables a considerar y la relación entre estas. Justifique
su diseño sea riguroso(a).

 Si su solución incluye el desarrollo de algun SW también debe realizar el diseño del mismo, de acuerdo a la pauta anterior.
   Extensión máxima sugerida de 15 páginas.


\subsection{Diseño de experimentos} \label{disenoexp}
Defina los  experimentos que debe realizar para responder sus preguntas de investigación.
Cada experimento  debe precisar objetivo, escenarios posibles,
variables involucradas, medidas con las que se trabajará, pre-experimentos si es
necesario, relación entre las variables (hipótesis), herramientas y pasos del análisis.

 En el caso que los experimentos están encadenados, incluya un diagrama de causalidad de
experimentos.  Extensión máxima sugerida de 15 páginas.

Obs: lea cuidadosamente los documentos sobre diseño de experimentos \cite{inv,dawson,fundibeq,extracto_dawson}. Siga la pauta mostrada en las ppt de diseño de experimentos \cite{extracto_dawson_ppt}.

\subsection{Conclusiones} \label{conclusiones}
Incluya análisis crítico, pertinencia del problema, solución propuesta, proyecciones
y estado de avance. Extensión máxima sugerida  2 páginas.


\section{Diseño para otros tipos  Trabajos de Título }

 Para TT distintos a los explicados anteriormente deberán considerar las secciones que  les acomoden, en  acuerdo con su profesor guía.
  Si su solución incluye el desarrollo de algun SW también debe realizar el diseño del mismo, de acuerdo a la pauta anterior.